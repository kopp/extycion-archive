\documentclass[a4paper,12pt,draft]{article}

%% Standard
\usepackage[ngerman]{babel} 
\usepackage[utf8]{inputenc}
\usepackage[T1]{fontenc}

%% Mathe
\usepackage{amsmath}
\usepackage{amssymb}
\usepackage{amsthm}
\usepackage{latexsym}

%% Aufzaehlungen
\usepackage{enumerate}

%% Bilder
\usepackage{subfigure}
\usepackage{graphicx}

%% Absaetze usw
\usepackage{multicol}   
% zu verwenden mit 
% \begin{multicols}{$$Spaltenanzahl$$} 
%  text...
% \end{multicols}

%\setlength{\parindent}{0pt}    %Absatz-Einrueckung
%\setlength{\parskip}{3pt}      %Absatz-Abstaende


%% Fusszeilen
\usepackage{fancyhdr}
\pagestyle{fancy}
\renewcommand{\headrulewidth}{0pt}
\renewcommand{\footrulewidth}{0.4pt}
\lfoot{\NAME : \TITEL}
\cfoot{}
\rfoot{\thepage}
\lhead{}
\chead{}
\rhead{}
%\setlength{\headheight}{15pt}


%% Links
\usepackage[colorlinks=true,linkcolor=black,citecolor=black,%
bookmarksnumbered=true,breaklinks=true,pdfstartview=FitH]{hyperref}

%% Eigene Kommandos
% Differenzialrechnung
\newcommand{\diff}{\ensuremath{\mathrm d}}
\newcommand{\dx}{\ensuremath{\mathrm dx}}
\newcommand{\dvx}{\ensuremath{\mathrm d \vec x}}

% Lineares
\newcommand{\Mat}[1]{\ensuremath{\mathbf{#1}}}
\newcommand{\Ten}[1]{\ensuremath{\mathcal{#1}}}
\newcommand{\Ve}[1]{\ensuremath{\vec{#1}}}
% Vektoren sind Fette buchstaben
\renewcommand{\vec}[1]{\ensuremath{\boldsymbol{#1}}}
% Vektoren sind fett und nicht kursiv
% \renewcommand{\vec}[1]{\ensuremath{\mathbf{#1}}}
\newcommand{\skp}[2]{\ensuremath{\langle #1 \,|\, #2 \, \rangle}}


% Euler
\newcommand{\e}{\ensuremath{\operatorname{e}}}
\newcommand{\E}{\ensuremath{\operatorname{e}}}
\newcommand{\ir}{\ensuremath{\operatorname{i}}}
\newcommand{\I}{\ensuremath{\operatorname{i}}}

% allg Mathe
\newcommand{\R}{\ensuremath{\mathbb{R}}}
\newcommand{\folgt}{\ensuremath{\Rightarrow}}
\newcommand{\gdw}{\ensuremath{\Leftrightarrow}}


% Formatierung
\newcommand{\abs}[0]{\bigskip\noindent}
\newcommand{\const}{\ensuremath{\text{\emph{const}}}}


% Umgebungen
\newtheorem{satz}{Satz}[section]
\newtheorem{defi}{Definition}[section]
\newtheorem{lemma}{Lemma}[section]






\begin{document}



\newcommand{\NAME}{Michael Kopp}
\newcommand{\FACH}{Physik am Komputer}
\newcommand{\TITEL}{"Ubung 02}
\newcommand{\DATUM}{\today}


\pagestyle{plain} 
	% auskommentieren fuer fusszeile



%%%% Eigener Kopf

\sloppy

\begin{center}
\FACH
\hfill
\DATUM
\end{center}

\vspace{-5mm} % weniger abstand

\begin{center}
  \begin{Large}
 \textbf{\TITEL}
  \end{Large}
\end{center}

\vspace{-3mm}

\begin{center}
\hrulefill
%\quad
 %\raisebox{-1.5mm}{\NAME}
% \,
\quad 
\textit{\NAME}
\,
\hrulefill
\end{center}
 
 
%%%%%%%%%%%%%%%%%%%%%%%%%%%%%
%%%%%%%%%%%%%%%%%%%%%%%%%%%%%%
%%%%%%%%%%%%%%%%%%%%%%%%%%%%%%%%

\noindent

\paragraph{Aufgabe 1}
\label{sec:aufgabe_1}

Siehe \verb+fread03.cpp+.




\paragraph{Aufgabe 2}
\label{sec:aufgabe_2}

\begin{itemize}
\item Beim initialisieren verwende ich 0 als Startwert -- damit sieht
  man, wenn am Schluss wieder 0 in der Variablen steht, dass etwas
  nicht geklappt hat.
\item Schleife die Maximalwerte ermittelt in \verb+fread04.cpp+.
\item Schleife die gr"o"sere Zahl nach unten tauscht in \verb+fread05.cpp+
\item Vollst"andiger Bubble-Sort: \verb+fread06.cpp+. Wichtig: Hier
  kann eine Ausgabedatei als drittes Argument angegeben werden. Ohne
  das wird in \verb+fread06_ausgabe+ ausgegeben.
\item \verb+sort data > foo+ ausgef"uhrt;  
\verb+diff foo fread06_ausgabe+ zeigt \emph{nichts} an -- also keine
  Unterschiede.\footnote{Um das zu testen habe ich das Programm noch
    ohne Parameter laufen lassen woraufhin 250 Messwerte gez"ahlt
    wurden -- dann hat diff gemeckert... Bei dieser Gelegenheit
  musste ich auch das Error-handeling neu schreiben; bei meinem
  gcc scheint atoi ein NULL zur"uckzugeben, wenn
  keine Zahl eingegeben wird.}
\item F"ur die erste eingelesene Date (insg. $n$) werden $n$
  Durchl"aufe gemacht, f"ur die zweite $n-1$, f"ur die $i$-te
  $n-i+1$. F"ur $n$ Zahlen braucht man also
  \begin{equation}
    \label{eq:1}
    \sum_{i=1}^n n-i+1 = -{{n^2+n}\over{2}}+n^2+n 
  \end{equation}
  Durchl"aufe. Der Algorithmus skaliert damit mit $O(n^2)$.
\end{itemize}

 
 

\paragraph{Aufgabe 3}
\label{sec:aufgabe_3}

Der Quicksort findet sich in \verb+quick01.cpp+; hier wurde auf
Kontrollelemente verzichtet: Die Dateien m"ussen manuell im Quellcode
angegeben werden.

Das besondere ist, dass dieses Programm jeden Schritt ausgibt; diese
kann man verwenden, um die Arbeitsweise zu visualisieren
(vgl. Kommentare am Programmanfang). Dazu wurde die Datei
\verb+anim.flv+ erzeugt.

Der Quicksort braucht bei \verb+data+ f"ur \emph{alle} Daten ca. $9$
sekunden (das kann man ermitteln indem man
\verb+date && ./a.out > foo && date+
ablaufen l"asst und die Sekunden bei \verb+date+ vergleicht. Der
Bubble-Sort hat daf"ur wesentlich l"anger gebraucht.

Der Grund liegt darin, dass quicksort im Idealfall mit $O(n \, \log
n)$ arbeitet.
 
 
 
 
 
 



\end{document}














%%% Local Variables: 
%%% mode: latex
%%% TeX-master: t
%%% End: 
