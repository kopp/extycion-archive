\documentclass[a4paper,12pt]{article}

%% Standard
\usepackage[ngerman]{babel} 
\usepackage[utf8]{inputenc}
\usepackage[T1]{fontenc}

%% Mathe
\usepackage{amsmath}
\usepackage{amssymb}
\usepackage{amsthm}
\usepackage{latexsym}

%% Aufzaehlungen
\usepackage{enumerate}

%% Bilder
\usepackage{subfigure}
\usepackage{graphicx}

%% Absaetze usw
\usepackage{multicol}   
% zu verwenden mit 
% \begin{multicols}{$$Spaltenanzahl$$} 
%  text...
% \end{multicols}

%\setlength{\parindent}{0pt}    %Absatz-Einrueckung
%\setlength{\parskip}{3pt}      %Absatz-Abstaende


%% Fusszeilen
\usepackage{fancyhdr}
\pagestyle{fancy}
\renewcommand{\headrulewidth}{0pt}
\renewcommand{\footrulewidth}{0.4pt}
\lfoot{\NAME : \TITEL}
\cfoot{}
\rfoot{\thepage}
\lhead{}
\chead{}
\rhead{}
%\setlength{\headheight}{15pt}


%% Links
\usepackage[colorlinks=true,linkcolor=black,citecolor=black,%
bookmarksnumbered=true,breaklinks=true,pdfstartview=FitH]{hyperref}

%% Eigene Kommandos
% Differenzialrechnung
\newcommand{\diff}{\ensuremath{\mathrm d}}
\newcommand{\dx}{\ensuremath{\mathrm dx}}
\newcommand{\dvx}{\ensuremath{\mathrm d \vec x}}

% Lineares
\newcommand{\Mat}[1]{\ensuremath{\mathbf{#1}}}
\newcommand{\Ten}[1]{\ensuremath{\mathcal{#1}}}
\newcommand{\Ve}[1]{\ensuremath{\vec{#1}}}
% Vektoren sind Fette buchstaben
\renewcommand{\vec}[1]{\ensuremath{\boldsymbol{#1}}}
% Vektoren sind fett und nicht kursiv
% \renewcommand{\vec}[1]{\ensuremath{\mathbf{#1}}}
\newcommand{\skp}[2]{\ensuremath{\langle #1 \,|\, #2 \, \rangle}}


% Euler
\newcommand{\e}{\ensuremath{\operatorname{e}}}
\newcommand{\E}{\ensuremath{\operatorname{e}}}
\newcommand{\ir}{\ensuremath{\operatorname{i}}}
\newcommand{\I}{\ensuremath{\operatorname{i}}}

% allg Mathe
\newcommand{\R}{\ensuremath{\mathbb{R}}}
\newcommand{\folgt}{\ensuremath{\Rightarrow}}
\newcommand{\gdw}{\ensuremath{\Leftrightarrow}}


% Formatierung
\newcommand{\abs}[0]{\bigskip\noindent}
\newcommand{\const}{\ensuremath{\text{\emph{const}}}}
\newcommand{\bild}[1]{\texttt{ ($\rightarrow$ Bild #1)}}
\newcommand{\bildd}[2]{\texttt{ ($\rightarrow$ Bild #1,} #2 \texttt{ )}}


% Umgebungen
\newtheorem{satz}{Satz}[section]
\newtheorem{defi}{Definition}[section]
\newtheorem{lemma}{Lemma}[section]






\begin{document}



\newcommand{\NAME}{Michael Kopp}
\newcommand{\FACH}{Praktikum}
\newcommand{\TITEL}{Auswertung von A20}
\newcommand{\DATUM}{\today}


\pagestyle{plain} 
	% auskommentieren fuer fusszeile



%%%% Eigener Kopf

\sloppy

\begin{center}
\FACH
\hfill
\DATUM
\end{center}

\vspace{-5mm} % weniger abstand

\begin{center}
  \begin{Large}
 \textbf{\TITEL}
  \end{Large}
\end{center}

\vspace{-3mm}

\begin{center}
\hrulefill
%\quad
 %\raisebox{-1.5mm}{\NAME}
% \,
\quad 
\textit{\NAME}
\,
\hrulefill
\end{center}
 
 
%%%%%%%%%%%%%%%%%%%%%%%%%%%%%
%%%%%%%%%%%%%%%%%%%%%%%%%%%%%%
%%%%%%%%%%%%%%%%%%%%%%%%%%%%%%%%

\noindent


\section*{Formeln}
\label{sec:formeln-1}

Die Energiedifferenz $\Delta E$ zwischen zwei Niveaus im Atom wird
durch ein Photon der Frequenz $\nu$ bzw. $\omega$ ausgeglichen; es
gilt der Zusammenhang
\begin{equation}
  \label{eq:1}
  \Delta  E = \omega \, \hbar = \nu \, h
\end{equation}
mit dem Planck'schen Wirkungsquantum $h$ und $\hbar = h/2\pi$.

F"ur die Wellenl"ange gilt
\begin{equation}
  \label{eq:4}
  \lambda = \frac{c}{\nu}
\end{equation}
mit der Lichtgeschwindigkeit $c$ und der Frequenz $\nu$.

Mit \eqref{eq:1} und \eqref{eq:4} folgt
\begin{equation}
  \label{eq:5}
  \lambda = \frac{h \, c}{\Delta  E} \;.
\end{equation}




\section*{Auswertung}
\label{sec:auswertung}


\subsection*{Kurvendiskussion}
\label{sec:kurvendiskussion}


Bei den beiliegenden Diagrammen auf der $x$-Achse stets die
Beschleunigungsspannung $U_1$ in Volt und auf der $y$-Achse der Strom
$I$ in Nano-Amp\'ere angegeben; alle weiteren Parameter sind im
Diagramm gegeben.

\abs
Die Kurve, die in den Diagrammen mit $\Gamma$ bezeichnet ist, ist
vermutlich nicht aussagekr"aftig: Sie passt in allen Diagrammen
deutlich nicht in ein logisches Schema. Dies war die Kurve der
allerersten Messung beim Hg-Aufbau; m"oglicherweise war die Maschine
noch nicht richtig eingelaufen und so wurde diese "`falsche"'
Linie produziert.


\paragraph{Allgemeines}
\label{sec:allgemeines}

Die Maxima und Minima der Kurven scheinen stets mit wachsender
Beschleunigungsspannung \emph{exponentiell} anzusteigen \bild 0, wobei
kleine Spannungen systematisch nicht dazu passen wollen. Dieser
Verlauf stimmt mit dem Verlauf einer Kennlinie bei einer
Glimmentladung "uberein. In diesem Sinne k"onnte der exponentielle
Anstieg auf zunehmende Ionisierung hinweisen, wie sie sich bei der
Plasmabildung einer Glimmentladung einstellt. 

Der exponentielle Anstieg von sowohl Minima als auch Maxima l"asst
sich (auch) dadurch erkl"aren, dass die Elektronen nach $k$
inelastischen St"o"sen noch eine gewissen Strecke zur"ucklegen, auf
denen sie beschleunigt werden, aber nicht mehr sto"sen. Mit steigender
Beschleunigungsspannung nimmt  die Zahl $k$ zu und besagte freie
Laufl"ange nimmt ab. Dadurch, dass die freie Strecke immer k"urzer
wird, sind hier weniger Atome, mit denen die Elektronen sto"sen
k"onnten, und dadurch erreichen mehr den Detektor. Auch wenn ein
Elektron seine $k$ St"o"se noch nicht absolviert hat -- also
eigentlich noch einen machen m"usste -- sinkt so mit steigender
Spannung die Wahrscheinlichkeit, diesen letzten Sto"s nachzuholen und
damit erreichen mehr Elektronen den Detektor mit weniger absolvierten
St"o"sen und sorgen f"ur einen "`Grundstrom"'.



\paragraph{Variation der Temperatur (Hg)}
\label{sec:variation_der_temperatur}

Erh"oht man die Temperatur des Hg-Aufbaus, vermindert sich der Strom
\bild 1.

Das liegt vermutlich daran, dass sich zwischen dem fl"ussigen und dem
gasf"ormigen Quecksilber ein gewisses Gleichgewicht einstellt, welches
sich f"ur h"ohere Temperaturen eher in Richtung gasf"ormig
verlagert\footnote{Prinzip von \textsc{Braun} und \textsc{Le
    Chatellier}}. Dadurch, dass mehr Gasteilchen vorhanden sind, nimmt
die Sto"swahrscheinlichkeit f"ur die Elektronen zu.

Man kann die Betrachtungen auch "uber die Mittlere freie Wegl"ange
$\lambda$ aufziehen: Ein Elektron wird so lange beschleunigt, bis es
die n"otige Energie $E_a$ f"ur eine Anregung hat -- doch dann kann es
diese nicht sofort abgebeb, sondern muss zuerst sto"sen. Dabei legt es
\emph{im Mittel} die Strecke $\lambda$ zur"uck. Ist $\lambda$ gro"s,
ist die Wahrscheinlichkeit, dass ein Elektron seine Energie durch
St"o"se verliert, kleiner und bei steigender Termperatur sinkt die
freie Wegl"ange.\footnote{Vermutlich nicht nur aufgrund des Mehr an
  Teilchen, sondern auch, weil die Atome sich auch st"arker
  (thermisch) bewegen.}

Weiter kann man beobachten, dass die Minima der einzelnen Linien nicht
genau "ubereinanderliegen -- und dass der Unterschied
temperaturabh"angig ist. Bei kleiner Beschleunigungsspannung sind die Minima der
"`kalten"' Linien st"arker gegen"uber den "`warmen"' Linien nach links
(weniger Spannung) verschoben als bei hoher Spannung \bildd 1 {blaue
  Striche}.

Dies k"onnte daran liegen, dass die Elektronen bei den tiefen
Temperaturen eine so gro"se freie Wegl"ange $\lambda$ zus"atzlich
beschleunigt werden, dass ihre Energie ausreicht, um ein n"achst
h"oheres Niveau als das Grundniveau anzuregen: Der "`normale"'
Zustandswechsel $6s \to 6p_0$ hat eine Energiedifferenz von
$4.67\operatorname{eV}$, der n"achst h"ohere $6s \to 6p_1$ die
Differenz $4.89\operatorname{eV}$.\footnote{Der n"achst, $6s \to 6p_3$
  h"atte $5.46\operatorname{eV}$, was ich schon f"ur zu hoch halte...}
Bei\footnote{Der Effekt tritt bei $30\operatorname{V}$ auf, ich
  sch"atze den Abstand Gitter-Kathode mit $0.75\operatorname{cm}$ ab.}  $E \sim 30\operatorname{V}/0.75\operatorname{cm}$ w"are dies bei
$\lambda \sim 55\operatorname{\mu m}$ erreicht.  Dadurch verlieren die
Elektronen mehr Energie, k"onnen also vergleichsweise weniger gut
gegen die Bremsspannung ankommen.

Bei gro"sen Spannungen
relativiert sich dies wieder, weil dann auch bei kleineren $\lambda$
durch das st"arkere E-Feld die h"ohere Anregungsenergie erreicht
werden kann.\footnote{Ganz allgemein sorgt ebendieser Effekt, dass
  Elektronen ein Mehr an Energie auf dem Weg $\lambda$ ansammeln
  k"onnen dazu, dass f"ur h"ohere Ordnungen die Abst"ande zwischen den
Maxima/Minima immer gr"o"ser werden \bild 2: Bei Ordnung $k$ hatten
die Elektronen $k$ mal die Gelegenheit, zus"atzliche Energie auf der
Strecke $\lambda$ zu sammeln und h"ohere Niveaus anzuregen. Mit der
Spannung steigt dadurch die Wahrscheinlichkeit, die h"oheren Zust"ande
anzuregen, was mehr Energie "`kostet"'.}






\paragraph{Variation der Heizspannung (Hg) bzw. Saugspannung (Ne)}
\label{sec:variation_der_heizspannung_hg}

Erh"oht man die Heitzspannung steigt der Strom \bild 3.

Die Erkl"arung d"urfte einfach die sein, dass mehr Elektronen aus der
Kathode ausgel"ost werden und so wegen Ladungserhaltung auch ankommen
m"ussen...

Auch hier verschieben sich wieder die Maxima gro"ser
Heitzspannungen f"ur kleine Beschleunigungsspannungen st"arker nach
links \bildd 3 {blaue Linien}, was damit begr"undet werden k"onnte, dass mehr Elektronen da
sind, die die freie Wegl"ange ausnutzen k"onenn, um Energie f"ur eine
h"ohere Anregung zu sammeln; es machen also mehr Elektronen eine etwas
h"oherenergetische Anregung und kommen so weniger gut gegen die
Bremsspannung an.

\abs
Den selben Effekt finden man auch bei der Ne-Anordnung \bild {5a}
(besser in \bild {5b}, wobei hier durch die Darstellung auf 2 Achsen
die "Ubersichtlichkeit nicht die beste ist ...) -- auch wenn es
hier keine Heitzspannung gibt, sondern eine Saugspannung, die aber den
selben Zweck erf"ullt; eine h"ohere Saugspannung sorgt f"ur mehr
Elektronen. Die Effekte sind also genau analog.



\paragraph{Variation der Bremsspannung (Hg und Ne)}
\label{sec:variation_der_bremsspannung_hg_und}

F"ur gr"o"sere Bremsspannung wird der Strom kleiner \bild 4, \bild 6. 

Das ist wieder einfach und logisch, weil weniger Elektronen die
Barriere "uberwinden k"onnen und registriert werden k"onnen. Die
Elektronen, die an die Anode kommen, sind in ihrer Energie
verteilt. Senkt man die Bremsspannung, erlaubt man es zus"atzlich den
Elektronen mit etwas weniger Energie, zu passieren. Deshalb ist auch
die Strom"anderung f"ur eine Beschleunigungsspannung und zwei
verschiedene Bremsspannungen abh"angig vom Absolutbetrag des Stromes
an der entsprechenden Stelle.

Interessant ist, wie beim Ne-Aufbau beim vorletzten sichtbaren Minimum
f"ur die maximale Bremsspannung die Strom-Intensit"at wirklich auf $0$
absinkt, f"ur geringere Bremsspannungen jedoch stets deutlich positiv
bleibt; genau hier beginnt der unter "`Allgemeines"' besprochene Effekt.



\abs
Zus"atzlich sieht man noch sch"on, wie die Schaubilder um genau die
Spannung auf der Abszisse verschoben werden, um die die Bremsspannung
ge"andert wird \bild 7, einfach weil wenn die Bremsspannung um
$0.5\operatorname{V}$ erh"oht wird, die Elektronen noch
$0.5\operatorname{V}$ mehr Beschleunigungsspannung brauchen, um das
selbe Intensit"atsniveau zu erreichen.






\subsection*{Anregungsenergien}
\label{sec:anregungsenergien}

Wie man in \bild 2 sieht, ist es eigentlich wenig sinnvoll, die
Anregungsenergie als Mittelwert der Abst"ande aller bestimmten Maxima
zu nehmen...

Tut man es dennoch, bekommt man, wenn man f"ur eine Linie den
Mittelwert der
Abst"ande der Maxima bildet und aus diesen Mittelwerten wiederum einen
Mittelwert:
\begin{equation}
  \label{eq:2}
  E_a^\text{Hg} = (4.89 \pm 0.22) \operatorname{eV} \;.
\end{equation}
Der Literaturwert\footnote{Haken Wolf} betr"agt $4.67\operatorname{eV}$...

Auf die gleiche weise erh"alt man
\begin{equation}
  \label{eq:3}
  E_a^\text{Ne} = (17.91 \pm 1.10)\operatorname{eV} \;.
\end{equation}
Der Literaturwert\footnote{C.E. Moore: Atomic Energy Levels} betr"agt
$16.6\operatorname{eV}$ f"ur die erste Anregung.



\subsection*{Anregungswellenl"angen}
\label{sec:anregungswellenlangen}

Die Energien $E_a^\text{Hg}$ und $E_a^\text{Ne}$ entsprechen den
Wellenl"angen\footnote{Aus der Fehlerfortpflanzung $\Delta f
  = \partial_i f \Delta x_i$ ergibt sich $\Delta \lambda = \left |
    \frac{h\,c}{E^2} \right | \, \Delta E$.}
\begin{equation}
  \label{eq:6}
  \lambda_a^\text{Hg} = (253 \pm 11.4) \operatorname{nm}
\end{equation}
und
\begin{equation}
  \label{eq:7}
  \lambda_a^\text{Ne} = (69,2 \pm 4.25) \operatorname{nm} \;.
\end{equation}







\subsection*{Leuchtschichten}
\label{sec:leuchtschichten}


Im Ne-Aufbau bilden sich orange leuchtende Schichten senkrecht zum Elektronenfluss aus: Steigert man die Beschleunigungsspannung, so bildet sich eine Leuchtschicht knapp vor dem zweiten Gitter (also vor der Anode). Dieses wandert beim Erh"ohen der Spannung weiter in Richtung Elektronenquelle, bis es die Mitte zwischen den beiden Gittern erreicht hat. Wenn das passiert, bildet sich sofort eine zweite Schicht direkt am Gitter $G_2$, die auch wider wandert. Wenn sie ein Drittel des Gitterabstandes zur"uckgelegt hat, entsteht wieder eine Schicht und so weiter. 

Man muss jedoch vorsichtig sein: Wenn zu viele Schichten entstanden sind, so kann es zu einer Glimmentladung kommen: Alle Schichten vereinigen sich dann und man sieht nur ein helles oranges Leuchten in der R"ohre. Dies besch"adigt die Apperatur und der Elektronenstrom muss abgebrochen werden.

Zu erkl"aren sind diese Leuchterscheinungen dadurch, dass in diesen leuchtenden Zonen die Elektronen gerade ihren Sto"s ausf"uhren. Die verschiedenen Schichten bilden sich n"amlich immer genau dann, wenn die Spannung wieder einem vielfachen der Anregungsenergie entspricht. Man sieht hier auch, dass die Elektronen in der ersten Leuchtschicht genug Energie gesammelt haben um einmal zu sto"sen, dann beschleunigen m"ussen, um in der zweiten Leuchtschicht wieder sto"sen zu k"onnen usw.

Das orangene Licht hat eine Wellenl"ange in der Gr"o"senordnung von $\lambda^\text{orange} \sim 600-630 \operatorname{nm}$, unsere Messungen legen eine ungef"ahr 8.9 mal h"ohere Energiedifferenz f"ur den "Ubergang Grundzustand--Anregegter-Zustand nahe; es ist also nicht m"oglich, dass wir hier tats"achlich diesen "Ubergang sehen.

% Jedoch sieht man in \bildd {5b} {Bleistift} ein kleines, fast nicht
% aufgel"ostes Maximum mit einer Energie von $7.6\operatorname{eV}$, was einer
% Wellenl"ange von 


Die Anzahl der Schichten ist damit ungef"ahr gegeben durch\footnote{$\lfloor x \rfloor$ beschreibt die n"achst kleinere Ganzzahl}
\begin{equation}
	m = \lfloor \frac {U_1} {E_a} \rfloor
\end{equation}
und deren Position durch
\begin{equation}
	x = \frac L {U_1 / E_a} \cdot k \text { mit } k = 1,2, ..., m \;,
\end{equation}
wobei $L$ der Abstand der beiden Gitter ist.







\section*{Fehlerrechnung}
\label{sec:fehlerrechnung}


\subsection*{Anregungsenergien}
\label{sec:anregungsenergien-1}

Wie erwartet lagen beide ermittelten Durchschnittswerte zu hoch. In
\bild 8 sind die Maxima graphisch aufgetragen, zusammen mit einer
mutigen Ausgleichsgerade. Mit dieser kann man den Fehler
m"oglicherweise absch"atzen, den die Zunahme der Differenzen der
Maximaabszissen verursachen. Der Durchschnitt liegt irgendwo in der
Mitte des Punktehaufens -- die Gerade soll auch hier durchlaufen. Man
erh"alt also eine sehr grobe Absch"atzung f"ur den Fehler, wenn man sich
anschaut, wo die Gerade $n=1$ ber"uhrt. Mit der Steigung
$0.2\operatorname{eV}/5\text{Ordnungen}$ ergibt sich eine Differenz
von der Gr"o"senordnung $0.2\operatorname{eV}$ -- diese w"urde unser
Ergebnis n"aher an den Literaturwert r"ucken.




\abs
Bei Neon ist der Unterschied der Differenzen im Mittel bei
$1.51\operatorname{eV}$; da wir hier f"ur die Bestimmung der
Anregungsenergie die Differenzen der 3,4,5,... Sto"sordnungen
verwendet haben, erhalten wir eine Korrektur in der Gr"o"senordnung
$1.5\operatorname{eV}$. 

Hier macht eine Graphik keinen Sinn, weil wir meist nur drei oder vier
Maxima auswerten konnten; die Rechnung wurde in einer Tabelle angestellt.





\subsection*{Einfluss der Saugspannung}

Beim Ne-Aufbau ist auch dann ein Strom messbar, wenn dies eigentlich nicht
m"oglich sein sollte: In \bild 9 sieht man, dass wir die Bremsspannung auf
$U_2=7\operatorname V$ gesetzt haben, aber auch dann Elektronen detektiert
wurden, wenn die Beschleunigungsspannung unter dieser Spannung liegt \bildd 9
{Bleistift}. Bei kleinen Saugspannungen ist der Effekt klein, bei gro"sen
gro"s. Deswegen vermute ich, dass die Saugspannung die Elektronen schon
vorbeschleunigt -- und diese Vorbeschleunigung geht nicht in unsere Rechnungen
ein (auch wenn sie einmalig ist und bei h"oheren Ordnungen nur wenig ins
Gewicht fallen d"urften).

\abs
Dass die Saugspannung unabdingbar ist, sieht man jedoch in \bild {10}: Hier ist
der Strom vernachl"assigbar klein selbst f"ur die maximale
Beschleunigungsspannung, wenn nur die Saugspannung klein gew"ahlt wird. Ohne
Saugspannung w"are der Versuch damit nicht durchf"uhrbar.

Die (f"ur mich bisher) einleuchtendste Erkl"arung daf"ur ist, dass sich eine negative "`Ladungswolke"' hinter der Kathode bildet und die Elektronen sich nicht wirklich in die R"ohre bewegen, wo sie beschleunigt werden. Durch das Gitter $G_1$ wird diese Ladungswolke gezielt abgesaugt.





\section*{Zusammenfassung}


Wir konnten sch"on das Wechselspiel von beinahe "aquidistanten Minima und Maxima der Kurven beobachten -- jedoch auch, dass die Maxima eben nicht \emph{ganz genau} "aquidistant sind, sondern der Abstand linear mit der Ordnung der Maxima zunimmt, was vermutlich an der zus"atzlichen Beschleunigung auf der freien Wegl"ange liegt.

Wir konnten die Abh"angigkeit der Kurven von verschiedenen Versuchsparametern beobachten und mit unserem Modell begr"unden.

Als Anregungsenergien ergeben sich mit den Korrekturen aus der Fehlerrechnung
\begin{align*}
	E_a^\text{Hg} &= (4.69 \pm 0.42) \operatorname{eV}\\
	E_a^\text{Ne} &= (16.41 \pm 2.60)\operatorname{eV} \;,
\end{align*}
was eine relative Abweichung\footnote{$F(x) = (x-\bar x)/\bar x$ mit Literaturwert $\bar x$} von den Literaturwerten von 
\begin{align*}
	F( E_a^\text{Hg} ) &\approx 0.4 \% \\
	F( E_a^\text{Ne} ) &\approx -1.1 \% \\
\end{align*}
bedeutet.


\abs
Was ich an dem Versuch unheimlich reizvoll fand, war, dass er oberfl"achlich betrachtet sehr einfach ist, er aber gleichzeitig viel Raum f"ur tiefergehende "Uberlegungen und Detailanalysen bietet. Was leider weniger spannend war, ist die Versuchsdurchf"uhrung an sich...









 
 
 
 
 
 
 
 



\end{document}














%%% Local Variables: 
%%% mode: latex
%%% TeX-master: t
%%% End: 
