\documentclass[a4paper,draft]{article}

\input{./define}

\usepackage[ngerman]{babel} %%necessary so that emacs does the
                            %%`"'-Thing right


\begin{document}


\input{./titlepage} % titelseite

\section*{Vorwort}
\input{./vorwort}

\tableofcontents
\clearpage % eine Leerseite folgt

%%%%%%%%%% %%%%%%%%%% %%%%%%%%%% %%%%%%%%%% %%%%%%%%%% %%%%%%%%%% %%%%%%%%%% 
%%%%%%%%%% %%%%%%%%%% %%%%%%%%%% %%%%%%%%%% %%%%%%%%%% %%%%%%%%%% %%%%%%%%%% 
%%%%%%%%%% %%%%%%%%%% %%%%%%%%%% %%%%%%%%%% %%%%%%%%%% %%%%%%%%%% %%%%%%%%%% 


\section{Atomare Gr"o"sen}\label{sec:atomare_grosen}

\subsection{Einige Hausnummern (Zahlenwerte)}
\label{sec:einige_hausnummern_zahlenwerte}

Ein paar interessante Zahlen sind in Tab. \ref{tab:hausnummern}
zusammengefasst.\footnote{Quelle: Vorlesung.}
\begin{table}[h]
  \centering
  \begin{tabular}{l l}
    \toprule
\textbf{L"angen} \\
Gr"o"se des Atomkerns & $5 \cdot 10^{-15}\operatorname{m}$\\
Ansdehnung der H"ulle & $1 \cdot 10^{-10}\operatorname{m}$\\
\midrule
\textbf{Massen}\\
Elektron & $9.11 \cdot 10^{-31}\operatorname{kg}$\\
Proton & $1.67261 \cdot 10^{-27}\operatorname{kg}$\\
Neutron & $1.67482 \cdot 10^{-27}\operatorname{kg}$\\
\bottomrule
  \end{tabular}
  \caption{Einige atomare Gr"o"sen}
  \label{tab:hausnummern}
\end{table}


Um sich eine Vorstellung von \textbf{Energieskalen} zu machen,
betrachtet man:\footnote{Quelle: Vorlesung}
\begin{description}
\item[Elektrostatische WW] $E \sim -\frac{1}{4\pi \varepsilon_0}
  \frac{Z e^2}{a_0 / Z} \sim 4 \operatorname{eV}$
\item[Relativistische Effekte] Aus $E = \gamma m c^2$ folgt f"ur die
  \emph{kinetische Energie}: $T = (\gamma-1) m c^2$ und f"ur $v \ll c$
  mit Taylor: $T \sim \frac{1}{2}mv^2 + \frac{3}{8}\frac{mv^4}{c^2}$;
  diese Korrekturen spielen nur bei sehr gro"sen Energien eine Rolle.
\item[Feinstruktur: Spin-Bahn-WW] Das kreisende Elektron erzeugt ein
  B-Feld -- nach Biot-Savart $B = \frac{\mu_0}{4\pi} Z e (\vec v
  \times \vec r) / r^3$ und mit $\vec L = m \vec v \times \vec r$ ist
  $\vec B = - \frac{Z e \mu_0}{4\pi r^3 m_e}\vec L$ --, das Elektron
  selbst hat magn. Dipolmoment $\mu \sim \mu_B = \frac{e\hbar}{2m_e}$
  und damit ist die Kopplungsenergie $E \sim - \vec \mu \cdot \vec B =
  \frac{\mu_0}{4\pi} \frac{\mu_B^2}{a_0^3}\cdot Z^4$. Durch diese
  $Z^4$-Abh"angigkeit ist dies f"ur schwere Atome sehr wichtig; Zahlen
  sind im Bereich $10^{-4}\operatorname{eV}$ bis $10^{-1}\operatorname{eV}$.
\item[Hyperfeinstruktur: Kern-Bahn-WW] Das magnetische Moment des
  Kernes $\mu_K$ verh"alt sich wie $\mu_K / \mu_B \sim m_e/m_K$; damit
  ist die Kopplung wesentlich schw"acher (f"ur das H-Atom $E \sim
  \frac{1}{1836}\mu_B B_e$). So ergeben sich Gr"o"senordnungen
  $10^{-7}\operatorname{eV}$ bis $10^{-4}\operatorname{eV}$.
\end{description}





\subsection{Bestimmung atomarer Gr"o"sen}
\label{sec:bestimmung_atomarer_grosen}


to be done soon



\subsection{Endliche Breite von Spektrallinien}
\label{sec:endliche_breite_von_spektrallinien}

\begin{Wichtig}
  Eine Spektrallinie kann niemals die Breite $0$ haben.
\end{Wichtig}

\paragraph{Nat"urliche Linienbreite}
\label{sec:naturliche_linienbreite}


Dies kommt daher, dass die emittierte Welle Anf"angt und aufh"ort. Da
sich das Spektrum als Fouriertransformierte der Wellengleichung
ergibt, die Wellengleichung aber nicht mehr absolut periodisch ist
reicht ein Fourierkoeffizient einfach nicht aus... Das Ergebnis ist
eine (kontinuierliche) Funktion.

Modelliert man das Spektrum durch einen Hertz-Dipol mit Ladung $e$ udn
AMplitude $x_0$, der mit $\omega$ schwingt, so strahlt dieser die
Leistung $P = \omega^4 e^2 x_0^2 / (6\pi\varepsilon_0 c^3)$ ab. Die
gesamtenergie des (harmonischen) Dipoloszillators
ist\footnote{Entweder die kinetische Energie ist maximal und die
  potentielle ist verschwindet oder anderst herum} $E = mv_{max}^2/2 =
m\omega^2 x_02 /2$; damit ist nach der Zeit
\begin{equation}
  \label{eq:abklingzeit_dipol}
  \tau = \frac{E}{P} = \frac{m\omega^2x_0^2 \, 6 \pi \varepsilon_0
    c^3}{2 \, \omega^4 e^2 x_0^2}
=
\frac{3m \pi \varepsilon_0 c^3}{\omega^2 e^2} 
\end{equation}
die Energie abgegeben.

Hat man einen Harmonischen Oszillator mit 
\begin{equation*}
  m\ddot x + D \dot x + k x = 0
\end{equation*}
so folgt mit dem Ansatz $x = A \, \E^{\lambda t}$:
\begin{equation*}
  \lambda = -\delta \pm \I \cdot \sqrt{\omega_0^2 - \delta^2} \text{
    mit } \delta = \frac{D}{2m} \text{ und } \omega_0^2 = \frac{k}{m} \;.
\end{equation*}
In der L"osungsfunktion schl"agt sich das dann nieder, dass man einen
Sin/Cos-Term mit $\omega = \sqrt{\omega_0^2 - \delta^2}$ hat und einen
D"ampfungsterm mit $\E^{-\delta t}$. 
% Nach Definition ist in diesem
% Modell die Abklingdauer die Zeit, wann die Amplitude auf den
% $1/\E$-ten Teil abgefallen ist -- also $\tau = 1/\delta$ bzw. $\delta
% = 1/\tau$

F"uhrt man von der Schwingungsgleichung die Fouriertransformation
durch, so erh"alt man einen Term, dessen Realteil (wenn man die Welle
in komplexer Darstellung hat) \emph{proportional}\footnote{bspw
  ignoriere ich hier die $1/2\pi$ der Fourierkoeffizienten und die Dipolamplitude}
\begin{equation*}
  \Re \int_{0}^{\infty }{e^{t\,\left(-\delta+\I\,\omega_0-\I\,\omega\right)}
 \;dt} = 
\frac{\delta}{\delta^2 + (\omega_0 - \omega)^2} \;.
\end{equation*}
Diese Funktion\footnote{Dies ist eine \emph{Lorentzlinie}.} hat bei $\omega = \omega_0$ ihr Maximum von
$1/\delta$ und bei $\omega = \omega_0 \pm \delta$ den Wert
$\frac{1}{2} \, 1/\delta$ -- also die halbe H"ohe. Damit ergibt sich
eine \emph{Halbwertsbreite von $\Delta \omega = \delta$}.

\begin{Def}
  [Halbwertsbreite]
Nimmt eine Glockenkurve $f(x)$ ihr Maximum bei $x'$ an, und den Wert
$f(x')/2$ an den Stellen $x_1$ und $x_2$, so ist $|x_1-x'| = |x_2-x'|
=: \Delta x$ die Halbwertsbreite.
\end{Def}

Mit der Unsch"arferelation\footnote{In Exphysschreibweise} ist
$\Delta  E \cdot \Delta t \approx \hbar / 2$ und damit $\Delta
\omega \approx 1/(2\Delta t)$. Ist $\tau = \Delta t $ die mittlere
Lebensdauer folgt 
\begin{equation}
  \label{eq:12}
  \delta \approx \frac{1}{2 \tau}  
\end{equation}
 und damit f"ur die Halbwertsbreite $2\Delta \omega = 2\delta = \frac{1}{\tau}$.



\paragraph{Dopplereffekt}
\label{sec:dopplereffekt}

Ein weiterer Effekt ist der Dopplereffekt: Die Licht emittierenden
Elektronen bewegen sich auf den Betrachter zu oder von ihm weg,
dadurch wird ihr Licht ins Blaue bzw. ins Rote verschoben. Dabei
ver"andert sich die Frequenz nach
\begin{equation}
  \label{eq:dopplerfrequenz}
  \omega = \omega_0 \cdot (1 \pm v/c) \;.
\end{equation}

Setzt man nun f"ur die Geschwindigkeitsverteilung eine
Boltzmann-Verteilung an, so folgt f"ur die Besetzungsdichte von
Teilchen mit der Geschwindigkeit\footnote{die mittlere Klammer ist die
Normierung, der $\exp$-Teil ist die Boltzmann-Verteilung mit der
Kinetischen Energie $mv^2/2$} $v$:
\begin{equation*}
  n(v) \diff v = N \cdot \left ( \frac{m}{2\pi k T} \right )^{1/2}
  \cdot \exp\frac{-mv^2 / 2}{kT} 
\end{equation*}
mit der Gesamtteilchenzahl $N$. Setzt man hier
(\ref{eq:dopplerfrequenz}) nach $v$ aufgel"ost ein erh"alt man unter
der Annahme $I(\omega) \propto n(\omega)$ die Intensit"atsverteilung
\begin{equation*}
  I(\omega) \propto \exp \frac{-m  \left ( \frac{c\omega_0 -
        c\omega}{\omega_0} \right )^2}{2kT} \;.
\end{equation*}
Diese hat ihr Maximum ($1$) bei $\omega = \omega_0$ und nimmt $1/2$
bei $\omega = \pm \frac{ \omega_0 \sqrt{2\ln(2) \cdot k m T} \pm c
  m}{c m}$; damit ergibt sich aus der Differenz der beiden Werte:
\begin{equation}
  \label{eq:14}
  \Delta \omega = \frac{\omega_0}{c} \left ( 8\ln(2) \frac{kT}{m}
  \right )^{1/2} \;.
\end{equation}






\section{Bor'sches Atommodell}
\label{sec:borsches_atommodell}



  Die \textbf{Bohr'sche Postulate}:
  \begin{enumerate}[(i)]
  \item Elektronen strahlen beim Kreiseln keine Energie ab sondern
    nur, wenn sie eine Bahn wechseln.
    Die Abgabe erfolgt dann "uber eine Lichtwelle mit $E_n - E_{n'} -
    h \nu = \hbar \omega$.
  \item Korrespondenzprinzip: F"ur weit au"sen liegende Schalen geht
    die "`neue"' Mechanik in die "`gewohnte"' "uber.
  \end{enumerate}


Durch das zweite Postulat kann man die Energie der au"sen liegenden
Bahnen klassisch berechnen; dazu verwendet man die Coulomb-Anziehung
und setzt sie mit der Zentripetalkraft gleich:
\begin{equation}
  \label{eq:1}
  F = \frac{1}{4\pi\varepsilon_0} \frac{e^2}{r^2} = mr \omega^2 \;.
\end{equation}


Die \textbf{Bohr-Sommenfeld-Quantisierung} stellt eine gewisse
Erweiterung des eigentlichen -- noch klassischeren -- Bohr'schen
Atommodells dar:
\begin{equation}
  \label{eq:bohr-sommerfeld}
  \oint p \diff q = n \cdot h  ~ \text{ mit } n \in \mathbb N\;.
\end{equation}
Setzt man f"ur die Bewegung des Elektrons $\vec p = m \cdot \vec v$
ein und geht von einer konstanten Geschwindigkeit aus, wobei $\vec v$
stets tangential zur Kreisbahn auf der die Elektronen sich bewegen --
also $\vec v = v \cdot \vec e_\varphi$ -- steht, so erh"alt man (f"ur
den Fall dass die Elektronen sich auf einer Kreisbahn mit Radius $r$ bewegen)
\begin{equation*}
  \oint m \, \vec v \diff \vec q = \int_{0}^{2 \pi} m \, v \vec
  e_\varphi \cdot r \, \vec e_\varphi \diff \varphi 
=
m v r \cdot 2 \pi \;.
\end{equation*}
Setzt man dies in (\ref{eq:bohr-sommerfeld}) ein, erh"alt man
\begin{equation*}
  mvr = n \frac{h}{2\pi} = n \, \hbar \;.
\end{equation*}

Nun kann man noch den Drehimpuls $L$ mit der linken Seite
identifizieren und erh"alt als Bedingung aus der
Bohr-Sommerfeld-Quantisierung:
\begin{equation}
\label{eq:2}
  { L = n \cdot \hbar } ~ ~  \text{ mit } n \in \mathbb N \;.
\end{equation}

\abs
Aus \eqref{eq:1} und \eqref{eq:2} folgt nun (unter anderem)
\begin{equation}
  \label{eq:3}
\boxed{  E_n = \frac{1}{2} E_n^\text{pot} =\frac{1}{2} \frac{-e^4
    m}{(4\pi\varepsilon_0)^2 \hbar^2} \cdot \frac{1}{n^2} } \;.
\end{equation}
Zusammen mit dem ersten Bohr'schen Postulat kann man daraus elegant
die verallgemeinterte Lymann-Formel herleiten:
\begin{equation}
  \label{eq:4}
  \frac{1}{\lambda} = \bar \nu = R \cdot \left ( \frac{1}{n^2} -
    \frac{1}{{n'}^2}\right )  \;,
\end{equation}
wobei $R$ die \emph{Rydberg-Konstante} ist mit\footnote{Quelle: Wikipedia} $R = \frac{\mu
  e^4}{8\varepsilon_0^2 h^3 c}$. Diese ergibt sich aus \eqref{eq:3},
wobei die Elektronenmasse $m$ durch die relative Masser $\mu :=
\frac{m_e \cdot m_n}{m_e + m_k}$ ersetzt wurde.

\abs
Diese Ersetzung hat ihre Ursache darin, dass man bei der Verwendung
von $m$ die Mitbewegung des Kerns vernachl"assigt -- mit der
Relaitvmasse $\mu$ jedoch nicht.\footnote{Es l"asst sich in Strenge
  zeigen, dass man diese Ersetzung vornehmen kann; dazu f"urht man die
oben skizzierte Rechnung in relativkoordinaten wie man sie aus dem
Kepler-Problem kennt durch; dann sieht man, dass man $m$ durch $\mu$
ersetzen darf.}

Man unterscheidet die beiden Rydberg-Konstanten $R$ und $R_\infty$,
wobei letztere sich rechnerisch ergibt, wenn man die Kernmasse gegen
$\infty$ laufen l"asst; damit ist die Relativbewegung des Kerns
ignoriert. Die Umrechung ist damit
\begin{equation*}
  R = R_{\infty} \cdot \frac{\mu}{m_e} = R_\infty \cdot
  \frac{1}{1+m_e/m_n} \;.
\end{equation*}
F"ur Wasserstoff ist diese Korrektur $\approx 0.99946$ -- also im
vertretbaren Rahmen.

Verwendet man  $m_n = m_\text{Proton} + m_\text{Neutron}$, so erh"alt
man $R$ f"ur \emph{Deuterium}. Um die $R$ zu unterscheiden schreibt
man $R_H = R$ f"ur Wasserstoff, $R_D$ f"ur Deuterium usw.



\subsection{Wasserstoff"ahnliche Spektren}
\label{sec:wasserstoffahnliche_spektren}

Die oben durchgef"uhrte Rechnung gilt f"ur das Wasserstoffatom;
trotzdem kann man sie auch auf Systeme mit einem Elektron -- den
sog. \textsc{Wasserstoff"ahnlichen Systemen} -- anwenden.

Die bedeutendsten Unterschiede ergeben sich in der Kernladungszahl
$Z$, welche die Coulomb-Anziehung variieren
\begin{equation}
  \label{eq:5}
  F_C = \frac{1}{4\pi\varepsilon_0} \frac{Z e^2}{r^2} \;,
\end{equation}
wobei die Kernmasse keinen zu gro"sen Effekt bewirkt\footnote{Wenn die
Protonenmasse verdoppelt wird, "andert sich der Faktor $1/(1-m_e/m_n)$
nur in der vierten Nachkommastelle}, da die Effektive
Masse des Elektrons $\mu$ davon nur wenig abh"angt. Die
Kernladungszahl geht dagegen mit $Z^2$ in $R$ ein:
\begin{equation}
  \label{eq:6}
  R(\mu,Z) = \frac{\mu Z^2 e^4}{8 \varepsilon_0 h^3 c} \;.
\end{equation}

F"ur den Bahnradius kann man aus \eqref{eq:1} (durch (\ref{eq:5})
modifiziert) und \eqref{eq:2} die Gleichung 
\begin{equation}
  \label{eq:7}
  r_n = \frac{4\pi\varepsilon_0\hbar^2}{Ze^2 \mu} \cdot n^2
\end{equation}
ableiten.


\begin{Anw}
  [Myonen Atome]
Myonen Atome haben anstatt eines Elektrons ein Myon in der
Umlaufbahn. Dieses ist ca. 207 mal so schwer wie ein Elektron. Setzt
man Werte f"ur Elektronen und Myonen eines wasserstoff"ahnlichen
Systems ($Z=1, m_n = m_p$) in (\ref{eq:7}) ein, so erh"alt
man
\begin{align*}
r_1(e^-) & \approx 5.29 \cdot 10^{-11}\operatorname{m}\\
r_1(\mu) & \approx 2.84 \cdot 10^{-13}\operatorname{m} \;;
\end{align*}
die Myon-Atome haben also eine wesentlich kleinere
Ausdehnung. Vergleicht man dies mit Tab. \ref{tab:hausnummern}, so
sieht man dass das um zwei Gr"o"senordnungen kleiner als ein
gew"ohnliches Atom ist und schon fast in den Bereich der Ausdehnung
des Kerns geht.
\end{Anw}





\subsection{Einf"uhrung der $l$-Quantenzahl im Bohr'schen Modell}
\label{sec:einf_der_l_quant_ansch}


Aus dem Bohr'schen Modell war bisher nur $n$-Quantenzahl bekannt mit
der $E_n$, $r_n$ etc bestimmt werden konnten. In Experimenten konnt
man jedoch sehen, dass jede einzelne dieser Spektrallinien etwas
unscharf waren -- unscharf deshalb, weil statt einer einzigen sich
viele feine Linien nebeneinander zeigten.

\textsc{Sommerfeld} nahm nun an, dass die Elektronen sich im
Bohr'schen Keplerproblem nicht unbedingt auf Kreisen bewegen m"ussen
-- so wie sich die Planeten auf ihren Bahnen auch auf Ellipsen bewegen
k"onnen. Die alte Quantenzahl $n$ bestimmte so die "`Hauptachse"' der
Ellipse, die kleine Hauptachse wurde durch $k$ bestimmt, wobei $k$ so
gew"ahlt werden muss, dass der Drehimpuls $k \cdot \hbar$ ist. 

Die Energie $E_n$ der Bewegung h"angt nur von der Hauptachse $a_n$ ab,
die verschiedenen Bahnformen sind characterisiert durch die Kleine
Halbachse $b_{n, k}$.

Die Nebenquantenzahl $k$ wurde durch die Bahndrehimpulsquantenzahl $l$
mit $l = k-1$ ersetzt; es gilt der Zusammenhang mit dem Drehimpuls:
\begin{equation}
  \label{eqn-drehimpuls-quantenzahl}
  | \vec L | = \sqrt{ l \cdot (l+1) } \cdot \hbar \;.
\end{equation}
F"ur die $l$-Quantenzahlen haben sich folgende Trivialnamen
eingeb"urgert:
\begin{center}
\begin{tabular}[h]{l  l l l l l l}
  Quantenzahl $l$ & 0 & 1 & 2 & 3 & 4 & 5 \\
  Drehimpuls $|\vec L|$ & $0$ & $\sqrt 2 \hbar$ & $\sqrt 6 \hbar$ &
  $\sqrt{12} \hbar$ & $\sqrt{20} \hbar$ & $\sqrt{30} \hbar$\\
  Name & s & p & d & f & g & h
\end{tabular}
\end{center}

\abs
F"ur die eigentliche Entartung verantwortlich -- also weswegen
verschiedene Spektrallinien beobachtet werden, obwohl die Bahnen auch
als Ellipsen die selbe Energie haben -- sind \emph{relativistische
  Effekte}: In Kernn"ahe sind die Elektronen schneller  als
"`au"sen"' -- sie scheinen schwerer zu sein, beschreiben also einen
kleineren Radius. Je n"aher das Elektron dem Radius kommt, desto
st"arker ist dieser Effekt; deshalb sind die feinen Linien f"ur ein
$n$ leicht verschieden.





\subsection{Quantenspr"unge durch St"o"se}
\label{sec:quantensprunge_durch_stose}

Elektronen k"onnen ihre kinetische Energie ungequantelt durch
Sto"svorg"ange an Atome "ubergeben, die daraufhin angeregt werden. 

Nachgewiesen wurde das beim Versuch von \textsc{Lenard} und dem
\textsc{Frank-Hertz}-Versuch. 

\paragraph{Lenard-Versuch}
\label{sec:lenard_versuch}

Es werden Elektronen beschleunigt und in eine Gas-Probe geschickt,
dann aber wieder abgebremst. Die Abbremsung ist st"arker als
die Beschleunigung. Elektronen schaffen es damit nicht bis zur
gegen"uberliegenden, negativ geladenen Platte. Schafft es allerdings
ein Elektron ein Gasteilchen zu ionisieren, so wird dieses Positive
Ion an die negative Platte gezogen und es flie"st ein Strom.

Man beobachtet, dass dieser Strom erst flie"st, wenn die
Beschleunigungsspannung eine gewisse Mindestspannung
"uberschreitet. (Dieses Beschleunigungspotential bezeichnet man dann
auch als \emph{Ionisationspotential}.)

\paragraph{Frank-Hertz-Versuch}
\label{sec:lenard_versuch-1}

Hier ist der Aufbau "ahnlich, nur dass die Bremsspannung nicht so hoch
ist: Elektronen k"onnen im Normalfall an die gegen"uberliegende Platte
gelangen. Erreicht die Beschleunigungsspannung einen
characteristischen Wert $U_r$, so f"allt jedoch pl"otzlich der Strom
an der gegen"uberliegenden Platte, der die auftreffenden Elektronen
anzeigt, stark ab; ebenso bei $2 \cdot U_r$, $3 \cdot U_r$,
\dots. Dies liegt daran, dass die Elektronen bei einem unelastischen
sto"s die "`Energie"' $U_r$ an ein Gasteilchen abgeben k"onnen um
dieses anzuregen; haben sie genut kinetische Energie, k"onnen sie
mehrmals sto"sen. Der Strom Steigt also so lange wie die Elektronen
noch $E_{kin} < n \cdot eU_r$ mit $n \in \mathbb N$ haben.

\abs
Die so bestimmte Spannung $U_r$ findet man wieder wenn man sich die
optischen Spektren des Gases anschaut -- doch gibt es noch weitere
Energien, die \emph{nicht} im Optischen Spektrum auftauchen. Es gibt
also Anregungszust"ande, auf die ein Sto"s das Atom bringen kann, ein
Photon jedoch nicht. Das liegt daran, dass es f"ur optische
"Uberg"ange besondere \textbf{Auswahlregeln} gibt, welche "Uberg"ange
erlaubt sind und welche nicht; die Auswahlregeln f"ur die Anregung
durch St"o"se sind verschieden von diesen "`optischen"' Auswahlregeln.

\abs
Weiter interessant: Die kinetische Energie ist nicht gequantelt: Auch
wenn die Elektronen eine gr"o"sere Energie haben als die
Anregungsenergie, so k"onnen sie doch anregen. Bei Anregung durch
Lichtquanten ist jedoch \emph{genau} der Energiebetrag n"otig.











\section{Quantenmechanische Betrachtungen}
\label{sec:quantenmechanische_betrachtungen}


\subsection{Wasserstoffatom -- Skizze}
\label{sec:wasserstoffatom}

Um die Schr"odingergleichung\footnote{$H\psi = E\psi$} des mit dem
Wasserstoffatom verbunden Hamilton
\begin{equation}
  \label{eq:Hamilton_h-atom}
  H = \frac{p_e^2}{2m_e} + \frac{p_n^2}{2m_n} -
  \frac{1}{4\pi\varepsilon_0} \frac{Ze^2}{r}
\end{equation}
zu l"osen geht man in Schwerpunkts- und Relativkoordinaten "uber,
separiert Radial- und Winkelteil ab ($\psi(r,\vartheta,\varphi) = R(r)
\cdot \chi(\vartheta,\varphi)$) und l"ost den Radialteil mit der
Polynommethode, wobei der Winkelanteil die Kugelfl"achenfunktionen sind.


Als L"osungen\footnote{Quelle: Wikipedia: "`Wasserstoffatom"'} findet
man:
\begin{equation*}
  \Psi_{nlm}(r,\vartheta,\varphi) = R_{nl}(r) \cdot Y_{lm}(\vartheta,
  \varphi )  \;,
\end{equation*}
wobei der Radialteil beschrieben wird durch
\begin{equation*}
  R_{nl}(r) = \sqrt {{\left(\frac{2}{n
          a_0}\right)}^3\frac{(n-l-1)!}{2n[(n+l)!]} } e^{- \rho / 2}
  \rho^{l} L_{n-l-1}^{2l+1}(\rho) \;,
\end{equation*}
wobei $ \rho = {2r \over {n a_0}} $ mit dem Bohr-Radius $a_0$ und $
L_{n-l-1}^{2l+1}(\rho) $ die \emph{zugeordneten Laguerre-Polynome}
sind und $Y_{lm}(\vartheta, \varphi ) $ die oben angesprochenen
Kugelfl"achenfunktionen.

\paragraph{Kugelfl"achenfunktionen}
\label{sec:kugelflachenfunktionen}

Die Kugelfl"achenfunktionen sind definiert "uber ihre Eigenschaft,
Eigenfunktionen des Drehimpulsoperators $L$ (bzw. $\vec L$) zu sein:
\begin{align*}
  {\vec L}^2 \cdot Y_{lm} & \propto l(l+1)  \cdot Y_{lm} \\
  L_z \cdot Y_{lm} &\propto m \cdot Y_{lm} \;.
\end{align*}
Eine Allgemeine Formel ist
\begin{equation*}
  Y_{lm}(\vartheta,\varphi) = 
\frac{(-1)^{l+m}}{2^l l!}
\sqrt{ \frac{2l + 1}{4\pi} \frac{(l-m)!}{(l+m)!} } \cdot \E^{ \I m
  \varphi} \cdot (\sin\vartheta)^m \cdot 
\frac{\diff ^{l+m}}{\diff (\cos\vartheta)^{l+m} }
(\sin\vartheta)^{2l} \;.
\end{equation*}
Die ersten sind in Tab. \ref{tab:kugelflaechenfunktionen} zu finden.

\begin{table}[h]
  \centering
  \begin{tabular}{l | l l l l}
\toprule
    ~ & $l=0$ & $l=1$ & $l=2$ & $l=3$ \\
\midrule
$m=-3$ & & & & $\sqrt{\tfrac{35}{64 \pi}} \sin^{3}{\vartheta}\,e^{-3i
  \varphi}$\\
$m=-2$ & & & $\sqrt{\tfrac{15}{32 \pi}} \sin^{2}{\vartheta} \, e^{-2i
  \varphi}$ & $\sqrt{\tfrac{105}{32\pi}}
\sin^{2}{\vartheta}\cos{\vartheta}\,e^{-2i \varphi}$\\
$m=-1$ & & $\sqrt{\tfrac{3}{8 \pi}}   \sin{\vartheta} \, e^{-i
  \varphi}$ & $\sqrt{\tfrac{15}{8 \pi}}  \sin{\vartheta} \,
\cos{\vartheta} \, e^{-i \varphi}$ & $\sqrt{\tfrac{21}{64 \pi}}
\sin{\vartheta}\left( 5 \cos^{2}{\vartheta} - 1\right)\,e^{-i
  \varphi}$ \\
$m=0$ & $\sqrt{\tfrac{1}{4 \pi}}$ & $\sqrt{\tfrac{3}{4 \pi}}
\cos{\vartheta}$ & $\sqrt{\tfrac{5}{16 \pi}}  \left( 3
  \cos^{2}{\vartheta} - 1 \right)$ & $\sqrt{\tfrac{7}{16 \pi}}  \left(
  5 \cos^{3}{\vartheta} - 3 \cos{\vartheta}\right)$ \\
$m=1$ & & $-\sqrt{\tfrac{3}{8 \pi}}  \sin{\vartheta} \, e^{i \varphi}$
& $-\sqrt{\tfrac{15}{8 \pi}} \sin{\vartheta} \, \cos{\vartheta} \,
e^{i \varphi}$ & $-\sqrt{\tfrac{21}{64\pi}} \sin{\vartheta}\left( 5
  \cos^{2}{\vartheta} - 1\right)\,e^{i \varphi}$ \\
$m=2$ & & & $\sqrt{\tfrac{15}{32 \pi}}  \sin^{2}{\vartheta} \, e^{2i
  \varphi}$ & $\sqrt{\tfrac{105}{32 \pi}}
\sin^{2}{\vartheta}\cos{\vartheta}\,e^{2i \varphi}$\\
$m=3$ & & & & $-\sqrt{\tfrac{35}{64 \pi}} \sin^{3}{\vartheta}\,e^{3i \varphi}$\\
\bottomrule
  \end{tabular}
  \caption{Die ersten Kugelfl"achenfunktionen -- Quelle: Wikipedia}
  \label{tab:kugelflaechenfunktionen}
\end{table}

Wichtige Eigenschaften der Kugelfl"achenfunktionen sind
\begin{itemize}
\item Orthonormalit"at
\item Parit"at: F"ur $\vec r \mapsto - \vec r$ gilt $r \mapsto r,
  \vartheta \mapsto \pi - \vartheta, \varphi \mapsto \pi +
  \varphi$. Die Kugelfl"achenfunktionen verhalten sich dann:
  $Y_{lm}(\pi-\vartheta,\pi+\varphi) = (-1)^l \cdot Y_{lm}(\vartheta,\varphi)$.
\item Konjugiertheit: Durch komplexes Konjugieren wird die $m$-Zahl
  gekehrt und die Funktion bekommt Parit"at $(-1)^m$: 
$\left[ Y_{l,m} \right ]^\ast = (-1)^m \cdot Y_{l,-m}$.
\end{itemize}


\paragraph{Zugeordnete Laguerre-Polynome}
\label{sec:zugeordnete_laguerre_polynome}

Die Zugeordneten Laguerre-Polynome\footnote{Die zugeordneten
  Laguerre-Polynome hängen mit den gewöhnlichen Laguerre-Polynomen
  über$$L_n^k(x) = (-1)^k \, \frac{{\rm d}^k}{{\rm d}x^k} \,
  L_{n+k}(x)$$ zusammen; es gilt die
  \emph{Rodrigues-Formel} $$L_n(x):=\frac{e^x}{n!} \frac{{\rm
      d}^n}{{\rm d}x^n}(x^n e^{-x}) \;.$$} erh"alt man "uber die
\emph{Rodrigues-Formel}:
\begin{equation*}
  L_n^k(x) = \frac{e^x \, x^{-k}}{n!} \, \frac{{\rm d}^n}{{\rm d}x^n}
  \, (e^{-x}\,x^{n+k}) \;;
\end{equation*}
die ersten lauten:
\begin{align*}
L_0^k(x) &= 1\\
L_1^k(x) &= -x + k + 1\\
L_2^k(x) &= \frac{1}{2}\,\left[x^2 - 2\,(k+2)\,x + (k+1)(k+2)\right]\\
L_3^k(x) &= \frac{1}{6}\,\left[-x^3 +3\,(k+3)\,x^2 -
  3\,(k+2)\,(k+3)\,x + (k+1)\,(k+2)\,(k+3)\right] \;.
\end{align*}











\subsection{Kopplung mit dem Elektrischen Feld}
\label{sec:kopplung_mit_dem_elektrischen_feld}


\subsubsection{Der Stark-Effekt}
\label{sec:der_stark_effekt}

\begin{Erfahrung}
Die Spektrallinien von Atomen in einem elektrischen Feld  spalten
sich auf.
\end{Erfahrung}

Im Wasserstoffatom und wasserstoff"ahnlichen Systemen ist diese
Aufspaltung f"ur Terme mit $l \neq 0$ ist proportional zu  $\| \vec
E\| = E$. Dieser \emph{lineare} Stark-Effekt kommt dann vor, wenn die
$l$-Entartung\footnote{Mehrere Zust"ande mit der selben $n$, aber
  verschiedenen $l$-Quantenzahlen haben die selbe Energie} nicht
aufgehoben ist.

Ist die $l$-Entartung dagegen schon ohne das E-Feld aufgehoben, so ist
die Verschiebung und Aufspaltung der Terme proportional zu $E^2$; der
\emph{quadratische Stark-Effekt}. 

Dieser ist anschaulich erkl"arbar, weil in einem Atom ein
elektrischesDipolmoment $\vec p = \Ten \alpha \cdot \vec E$ induziert
wird ($\Ten \alpha$ ist die Polarisierbarkeit\footnote{Bei gro"sen
  Atomen (oder gro"sem $n$) ist der Effekt besonders gut bemerkbar, da
  diese einfacher polarisierbar sind.  }) und dieses wiederum ein
Wechselwirkungspotential\footnote{$V = \int \vec p \cdot \diff \vec E
  = \frac{1}{2} \alpha E^2$} $V \propto \vec p \cdot \vec E = \Ten \alpha
\cdot E^2 \propto E^2$ hat.

Der lineare Stark-Effekt ist dagegen nur quantentheoretisch zu
verstehen.




\subsubsection{Quanthentheorie des linearen Stark-Effekts}
\label{sec:quanth_des_line_stark_effekts}




\subsubsection{Quantentheorie des quadratischen Stark-Effekts}
\label{sec:quant_des_quadr_stark_effekts}


Der Hamiltonoperator $H^0$ des Atoms  wird durch
das E-Feld gest"ort; diese St"orung macht sich als Term $H^S$
bemerkbar, der sich aus dem Wechselwirkungspotential zwischen Dipol
$e\vec r$ und dem E-Feld\footnote{Die Schreibweise dient das E-Feld
  von der Energie $E$ abzugrenzen} $\vec {\mathcal E}$ ergibt:
\begin{equation} 
\label{eq:8}
  H = H^0 + H^S = H^0 + (- \, e\vec r \cdot \vec {\mathcal E} ) =: H^0
  + \lambda H_1 \;,
\end{equation}
wobei $\lambda$ proportional zu $\mathcal E = \|\vec {\mathcal E}\|$
ist; damit k"onnen wir nun St"orungstheorie machen. Wichtig ist dabei,
dass beim quadratischen Start-Effekt die Atome \emph{nicht entartet}
sind; dadurch wird die St"orungsrechnung "`menschlich"'. Wir
entwickeln eine L"osung $\psi$ von $H\psi = E\psi$ als Potenzreihe in
$\lambda$ 
\begin{equation*}
  \psi = \psi_0 + \lambda^1 \psi_1 + \lambda^2 \psi_2 + ... \; \text{
    und }
E = E_0 + \lambda^1 E_1 + \lambda^2 E_2 + ... \;,
\end{equation*}
und setzen dies in (\ref{eq:8}) ein. Dabei gilt zu beachten, dass
$\psi_0$ eine ungest"orte L"osung sein muss: $\psi_0 = \phi_n^0$ (f"ur
ein bestimmtes $n$ und $H^0\phi_n^0 = E^0_n\phi_n^0$). Setzt man dies
in (\ref{eq:8}) ein dann erh"alt man eine Polynomgleichung in
$\lambda$, deren Koeffizienten man vergleichen kann -- also erh"alt
man f"ur jede Ordnung von $\lambda$ eine Gl. Auf diese wendet man  $\langle
\psi_0|$ an, und kann
mit der Bedingung\footnote{Diese darf man annehmen, weil man die
  Gleichungen auf der Linken Seite stets nach $(H^0 -
  E^0)|\psi_i\rangle = ...$ aufl"osen kann und hier eine Addition von
  $\psi_0$ keine Ver"anderung bringt.} $\langle \psi_0 \,|\,\psi_i
\rangle = \delta_{i0}$ weiter vereinfachen. Um $E_i$ zu erhalten,
wendet man $\langle \psi_0|$ auf die Gleichung von $\lambda^i$ an:
\begin{equation*}
  E_i = \langle \psi_0 | H^S | \psi_{i-1}  \rangle \;.
\end{equation*}
Um $\psi_i$ zu erhalten, entwickelt man es in\footnote{die $n$-te
  Komponente kommt nicht mit, weil $\langle \psi_0 \,|\,\psi_i
\rangle = \delta_{i0}$} $\{\phi_l^0\}_{l\neq n}$ und
wendet $\langle \phi_l|$ an:
\begin{equation*}
  \langle \psi_l \,|\, \psi_1 \rangle = \frac{\langle \psi_l | H^S | \psi_0
  \rangle}{E_l-E_n}
\end{equation*}



% Mit (\ref{eq:8}) k"onnen wir nun "`St"orungstheorie"' machen: Mit
% $\phi^0_n$ sind L"osungen von $H^0$ bezeichnet; wir entwickeln eine 
% L"osung $\psi$ von $H\psi = E\psi$ in $\{\phi^0_n\}_n$:
% \begin{equation}
%   \label{eq:9}
%   |\psi\rangle 
% = 
% \sum_{n} \langle \phi^0_n | \psi \rangle \,| \phi^0_n \rangle 
% \;; 
% \end{equation}
% wendet man dies auf (\ref{eq:8}) an, so kann man (der Linearit"at sei
% Dank) f"ur jedes $H^0 |\phi^0_n \rangle = E^0_n |\phi^0_n\rangle$
% schreiben; wendet man nun $\langle \phi_\nu^0|$ auf die Gl. an,
% erh"alt man -- durch die Orthonormalit"at der $\phi_n^0$:
% \begin{equation}
%   \label{eq:10}
%   E_\nu^0 \langle \phi_\nu^0 | \psi \rangle + \sum_{n} \langle
%   \phi^0_n | \psi \rangle \langle \phi_\nu^0 | H^S | \phi^0_n \rangle 
% =
% E \langle \phi_\nu^0 | \psi \rangle \;.
% \end{equation}

% F"ur die eigentliche St"orungstheorie nehmen wir nun an, dass unsere
% L"osung $\psi$ in $\lambda$ entwickelbar ist. F"ur $\lambda = 0$ muss
% $\psi = \phi_\mu^0$ sein (f"ur ein bestimmtes $\mu$) 








\subsection{Licht-Atom Wechselwirkung}
\label{sec:lichat_atom_wechselwirkung}


\subsubsection{Wechselwirkung Zwei-Atom-System mit koh"arentem resonanten Lichtfeld}
\label{sec:wechselwirkung_zwei_atom_system_mit}

Wir modellieren unser Atom als System mit nur zwei
Zust"anden. D.h. wir nehmen an, dass alle weiteren Zust"ande durch das
betrachtete Lichtfeld nicht angeregt werden k"onnen.

Der Hamilton des Systems setzt sich zusammen aus dem f"ur das Atom
$H^0$ und einem St"orfaktor $H^S$. Das E-Feld des Lichts ist
\begin{equation*}
  \vec {\mathcal E} = \vec {\mathcal E}_0 \cos(\vec k \cdot \vec r -
  \omega t) \;.
\end{equation*}
Wir nehmen weiter an, dass das Atom sich bei $\vec r = \vec 0$
befindet. 

Die Lichtwellenl"ange ($\sim 10^{-6}\operatorname{m}$) ist sehr gro"s
im Vergleich zum Atom ($\sim 10^{-10}\operatorname{m}$) -- damit darf
man die Ortsabh"angigkeit von $\vec{\mathcal E}$ getrost
vernachl"assigen, au"serdem sei o.B.d.A. $\vec {\mathcal E}_0 =
\mathcal E_0 \cdot \vec e_z$.

F"ur die Lichtwelle ist das Atom ein Dipol; das Potential der
Wechselwirkung ist $V^S = - \vec p \cdot \vec {\mathcal E}$ und $\vec
p = e \vec r$; damit
\begin{equation}
  \label{eq:11}
  V^S = H^S = - e z \mathcal E_0 \cos(\omega t) \;.
\end{equation}









\section{Laser}
\label{sec:laser}


Wichtige Eigenschaften des Lasers:\footnote{Zahlen aus Haken Wolf} (a)
hohe Koh"arenzl"ange ($\sim 300 000\operatorname{km}$) -- damit hohe
zeitliche Koh"arenz (b) starke B"undelung des Lichts (c) hohe
Strahlungsintensit"at (d) es ist m"oglich, Lichtimpulse bis $\sim
10^{-13}\operatorname{s}$ zu erzeugen.






\subsection{Aufbau und Funktionsweise}
\label{sec:aufbau_und_funktionsweise}



Atome eines aktiven Mediums werden durch Photonen ($E_1 \to E_2$) angeregt
("`\emph{Pumpen}"').  Diese relaxieren (zuf"allig) durch
\emph{Spontane Emission} ($E_2 \to E_1$) und strahlen ein Photon aus.
Diese Photonen regen andere Atome durch \emph{Stimulierte Emission}
an, Photonen mit genau den selben Eigenschaften (gleiche Frequenz,
Amplitude, Richtung) abzugeben.

Um einen Verst"arkung dieser Photonenlawine zu erhaltne, sind an zwei
Seiten des aktiven Materials Spiegel angebracht -- einer davon
halbdurchl"assig. Durch diese Spiegel k"onnen sich in dem Laser nur
Stehende Wellen mit bestimmten Wellenl"angen ($k \cdot \lambda/2 = L$
mit $k \in \mathbb N$, $L$: Abstand der Spiegel) aufhalten. Au"serdem
werden diejenigen Photonen, die nicht parallel zur Normalen der beiden
Spiegel stehen, nicht mehr reflektiert und entweichen.

Damit nun st"andig Photonen durch stimulierte Emission abgegeben
werden k"onnen, muss das Niveau $E_2$ st"andig st"arker besetzt sein,
als $E_1$; dann ist Emission h"aufiger als Absorption -- schlie"slich
sollen die einmal freigesetzten Photonen nicht gleich wieder
absorbiert werden.

Die Photonen, mit denen diese st"andige Anregung erfolgen soll,
sollten aber nicht direkt den gew"unschten Energiebetrag mitbringen,
weil sie sonst selbst Emissionen induzieren. Da das Pumpen von der
Seite geschieht, w"aren diese Photoenn nicht mehr auf der Gespiegelten
Linie zwischen den beiden Spiegeln -- w"urden also keinen gro"sen
Beitrag liefern. Stattdessen bringen die frisch eingebrachten Photonen
das System auf eine Energie $E_3$ ($E_1 < E_2 < E_3$); von dort regt
sich das System \emph{schnell} ab auf $E_2$ und ist so bereit f"ur
induzierte Emission.




\subsection{Beispiel: He-Ne-Laser}
\label{sec:he_ne_laser}

Zwischen zwei Elektroden findet eine Gasentladung statt. Diese
Elektronen regen durch Sto"s Helium auf $20.61$ oder $19.82$eV an. Die
He-Atome sto"sen gegen Neon und regen dieses vom Grudnzustand auf
"ahnliche Energien an ($20.66$ bzw. $19.78$eV) und von dort relaxiert
das Neon unter Aussendung von Licht auf verschiedene Niveaus
(induzierte Emission), wodurch Licht mit $\lambda \in \{ 3391.2,
1152.3, 632.8 \}\operatorname{nm}$ frei wird; Neon hat danach
mindestens noch $18.7\operatorname{eV}$, die es durch St"o"se wieder
abgibt.






\subsection{Laserbilanz}
\label{sec:laserbilanz}
















\subsection{Anwendung: Laserk"uhlung}
\label{sec:laserkuhlung}






























%%%%%%%%%% %%%%%%%%%% %%%%%%%%%% %%%%%%%%%% %%%%%%%%%% %%%%%%%%%% %%%%%%%%%% 
%%%%%%%%%% %%%%%%%%%% %%%%%%%%%% %%%%%%%%%% %%%%%%%%%% %%%%%%%%%% %%%%%%%%%% 
%%%%%%%%%% %%%%%%%%%% %%%%%%%%%% %%%%%%%%%% %%%%%%%%%% %%%%%%%%%% %%%%%%%%%% 





\end{document}



